	\documentclass[slidestop,uncompress,mathsans, 12pt]{beamer}
\usepackage[OT1]{eulervm}
%\usepackage[bars]{beamerthemetree}
\usepackage{xcolor}
\usepackage{graphicx}
\usepackage{array}
\usepackage{subfigure}
\usepackage[display]{texpower}
\usepackage{color}
%\usepackage{mathrsfs}
\usetheme{Frankfurt}
\setbeamercolor{alerted text}{fg=red}
%\beamertemplateshadingbackground{blue!5}{yellow!10}
\usecolortheme{whale}
\beamertemplateballitem
%\transglitter[direction=315]

%\useoutertheme{infolines}
%\title[Runzi Qin \space \space \space Rui Guo \space \space \space Yumeng He]{Qin Runzi}
%\subtitle{dsf}
%\institute{Shandong University}
%\date[march 10]{\today}
%\author[Runzi Qin]
\begin{document}
\begin{frame}
\title{The Fall of Hong Kong's Economy}
\subtitle{-----The Analysis of Reasons Behind}
\author{Runzi Qin\\   Hongmei Huang\\   Wanyun Hu\\ Jianing Zhang}
\institute{Shandong University}
\titlepage
\end{frame}

\section{Roadmap}
\subsection{Roadmap}
\begin{frame}
\frametitle{Roadmap}
\begin{itemize}
\item Introduction
\item Internal cause
\item External cause
\begin{enumerate}
\item  Return of Hong Kong
\item  Economic crisis
\item  The rise of mainland China
\end{enumerate}

\item  Conclusion
\end{itemize}
\end{frame}
\section{Internal causes}
\subsection{competitive}
\begin{frame}

\title{The Economic Competitiveness of Hong Kong}
\date{}
\titlepage
\end{frame}
\begin{frame}
\frametitle{The Economic Competitiveness of Hong Kong}
\begin{figure}[h]
\raggedleft
\includegraphics[width=1.1\textwidth]{hk4.jpg}
\label{threadsVsSync}
\end{figure}
\end{frame}

\begin{frame}
\frametitle{The Economic Competitiveness of Hong Kong}%[shrink]
The financial market of Hong Kong
And according to Levine (2000) and Arestis (1997), we use the M2/GDP, Loan/M2, Stock market value/GDP to present the competitiveness of Hong Kong financial market.\\
\begin{overprint}
\onslide<1>
\begin{figure}[h]
\centering
\includegraphics[width=1\textwidth]{hk10.png}
\label{threadsVsSync}
\end{figure}
\onslide<2>
\begin{figure}[h]
\centering
\includegraphics[width=1\textwidth]{hk11.png}
\label{threadsVsSync}
\end{figure}
\onslide<3>
\begin{figure}[h]
\centering
\includegraphics[width=1\textwidth]{hk12.png}
\label{threadsVsSync}
\end{figure}
\end{overprint}
\end{frame}
\begin{frame}
\frametitle{The Economic Competitiveness of Hong Kong}
Trade and logistics of Hong Kong\\
\begin{figure}[h]
\includegraphics[width=0.6\textwidth]{hk13.png}
\label{threadsVsSync}
\end{figure}
\begin{figure}[h]
\includegraphics[width=0.6\textwidth]{hk14.png}
\label{threadsVsSync}
\end{figure}
\end{frame}
\section{External causes}
\subsection{azi}
\begin{frame}
\title{Return to Hong Kong}
\date{}
\titlepage
\end{frame}
\begin{frame}
\frametitle{Return to Hong Kong}
\begin{itemize}
\item<2-> \alt<2>{\color{blue} Brain drain and capital flight
}{\color{gray} Brain drain and capital flight
}
\bigskip
\item<2-> \alt<3>{\color{blue} Confliction between Hong Kong and mainland
}{\color{gray} Confliction between Hong Kong and mainland
}
\end{itemize}
\end{frame}
\subsection{sdlkfj}
\begin{frame}
\frametitle{Return to Hong Kong}
Brain Drain and Capital Flight\\
\begin{definition}
\textcolor{cyan}{Brain drain}: a mass emigration of technically skilled people from a country or a region to another country. \\
\bigskip
\textcolor{magenta}{Capital flight}: assets or money rapidly flow out of a country or a region, due to an event of economic consequence. \\

\end{definition}
\end{frame}
\begin{frame}
\begin{figure}[h]
\centering
\includegraphics[width=0.8\textwidth]{hk15.jpg}
\label{threadsVsSync}
\end{figure}
\end{frame}
\begin{frame}{Return to Hong Kong}
 Why?
\begin{enumerate}
\item Uncertainty.
\bigskip
\item Better life in other countries.
\end{enumerate}
\end{frame}
\begin{frame}
\frametitle{Return to Hong Kong}
Brain Drain $\Longrightarrow$ Capital Flight
\setbeamercolor{uppercol}{fg=white,bg=blue}%
\setbeamercolor{lowercol}{fg=black,bg=white}%
\begin{beamerboxesrounded}[upper=uppercol,lower=lower col,shadow=true]{Process}
Many capitalists transferred their capital to other countries to avoid the uncertain of Hong Kong and people who emigrated brought their fortune abroad.\\
\end{beamerboxesrounded}
\end{frame}
\begin{frame}{Return to Hong Kong}
Effect
\begin{block}{}
\center{$Y=A \cdot K^a \cdot (EL)^{1-a}$}
\center{$\Downarrow$\\
\bigskip	Economic Downturn}
\end{block}
\end{frame}
\begin{frame}{Return to Hong Kong}
Conflict between Hong Kong and Mainland\\
\bigskip
\bigskip
Institutional difference.\\
\bigskip
\bigskip
Different economic circumstance 

\end{frame}
\subsection{Overview}
\begin{frame}
\title{The Influence of Economic Crisis on Hong Kong }
\date{}
\titlepage
\end{frame}
\begin{frame}[shrink]
\frametitle{The Influence of Economic Crisis on Hong Kong 
}
\begin{itemize}
\item Back	ground
\bigskip
\item Data
\bigskip
\item Explanation
\end{itemize}
\end{frame}
\pageTransitionSplitVI
\subsection{Background}
\begin{frame}

\frametitle{The Influence of Economic Crisis on Hong Kong }
Background\\
\bigskip
\transglitter[direction=315]
On July 2, 1997, Thailand gave up Fixed Exchange Rates which marked the beginning of Financial Crisis.\\
\end{frame}
\begin{frame}
\frametitle{The Influence of Economic Crisis on Hong Kong }
Background\\
\bigskip
On July 2, 1997, Thailand gave up Fixed Exchange Rates, having influence on some countries in Asia.\\
\bigskip
\transboxin<1>
From August 2007, Subprime mortgage crisis spreaded to the whole world which resulted from lending too much to those who are poor in credit.\\
\bigskip
Do they have influence on Hong Kong? Of course yes.
\end{frame}
\subsection{data}
\begin{frame}
\frametitle{The Influence of Economic Crisis on Hong Kong }
Data comes from Hong Kong SAR Government Census and Statistics Department.\\
\bigskip
\begin{figure}[h]
\raggedleft
\includegraphics[width=1\textwidth]{hk1.png}
\caption{Data}
\label{threadsVsSync}
\end{figure}
%\begin{table}[htbp]
%\centering  
%\begin{tabular}{ccccccccccccccccccc} 
%\hline
%&Series  %name&Unit&1995&1996&1997&1998&1999&2000&2001&2002&2003
 %&2004&2005&2006&2007&2008&2009&2010&2011\\ \hline  % \hline
% &GDP(\% real change pa)&\%&2.37&4.26&5.10&-5.88&2.51&7.06&0.56&1.66&3.06&8.70&
% &7.39&7.03&6.47&2.13&-2.46&6.77&4.82\\    
% \hline
%\end{tabular}
%\caption{Trade for the last victim: Gaizhen %Wang}
%\end{table}
\end{frame}

\begin{frame}
\frametitle{The Influence of Economic Crisis on Hong Kong }
The real GDP from 1995 to 2011.
\bigskip
\begin{figure}[h]
\raggedleft
\includegraphics[width=1\textwidth]{hk2.jpg}
\caption{The real GDP}
\label{threadsVsSync}
\end{figure}
\end{frame}

\begin{frame}{The Influence of Economic Crisis on Hong Kong }
GDP growth rate from 1995 to 2011 in Hong Kong.\\
\bigskip
\begin{figure}[h]
\raggedleft
\includegraphics[width=1\textwidth]{hk3.jpg}
\caption{GDP growth rate}
\label{threadsVsSync}
\end{figure}
\end{frame}

\subsection{Explanation}
\begin{frame}{The Influence of Economic Crisis on Hong Kong }
Explanation\\
\bigskip
\alert{Financial Crisis in 1997}
\bigskip

\begin{enumerate}[i]
\pause\item Thailand gave up Fixed Exchange Rates causing foreign exchange market chaotic.
\bigskip
\pause\item International speculators had two “attacks” in Hong Kong.
\bigskip
\pause\item A panic  in Financial market, and less investment.
\end{enumerate}

\end{frame}

\begin{frame}
\frametitle{The Influence of Economic Crisis on Hong Kong }
Explanation\\
\bigskip
\alert{Financial Crisis in 1997}
\bigskip
\begin{enumerate}[I]
\item <+-| alert@+> Subprime crisis. 
\bigskip
\item <+-| alert@+> Banks are less willing to lend. The world economy came into recession.
\bigskip
\item <+-| alert@+> Hong Kong being a part of world economy can't escape. 

\bigskip

\end{enumerate}
\end{frame}
\subsection{Huang}
\begin{frame}
\title {The Impact of The Rise of China's Economy on Hong Kong's Economic Status}
\date{}
\titlepage
\end{frame}
\begin{frame}{The rise of China's Economy}
Structure\\
\begin{itemize}
\item Hong Kong's special economic status for China\\
\item The Rise of China's economy\\
\item The impact on Hong Kong's economic status\\

\end{itemize}

\end{frame}
\begin{frame}
\frametitle{The rise of China's Economy}
The Special Economic Status of Hong Kong\\
\bigskip
\animate<1-3>%
\begin{itemize}[<+->]
\item The leading economic zone: “Dragon's Head”.
\item The mediation role on investment, capital introduction and the processing trade.
\item The financial center: provide offshore financial services.
\end{itemize}
\pause
\end{frame}

\begin{frame}
\frametitle{The rise of China's Economy}
The Reasons of the Special Status\\
\bigskip
\begin{enumerate}[A]
\item The high economic density.
\bigskip
\item The close economic relationship.
\bigskip
\item The multiple advantages of Hong Kong's economy.
\bigskip
\item The political reason.
\end{enumerate}
\end{frame}

\subsection{High Density}
\begin{frame}
\frametitle{The rise of China's Economy}
The High Economic Density\\
\bigskip
\bigskip
\begin{figure}[h]
\raggedleft
\includegraphics[width=1\textwidth]{hk5.jpg}
\label{threadsVsSync}
\end{figure}
~~~~~~~~~~~~~~~~~~~~~~~~~~~~~~~~~~~~~~~~~~~~~~~~Shanghai \$$3000$
\end{frame}

\subsection{close relationship}
\begin{frame}{The rise of China's Economy}
The Close Economic Relationship\\
\bigskip
The economic return runs before the political return. \\
\begin{overprint}
\onslide<1>

\begin{figure}[h!]
\centering
\includegraphics[width=1.05\textwidth]{hk6.jpg}
\label{threadsVsSync}
\end{figure}
\onslide<2>
Case: In 1980s,  the north movement of manufacture industries contributed to the economic development of Guangdong.\\ 
\begin{figure}[h!]
\centering
\includegraphics[width=0.7\textwidth]{hk7.jpg}
\label{threadsVsSync}
\end{figure}
\end{overprint}
\end{frame}

\begin{frame}
\frametitle{The rise of China's Economy}
Advantages of Hong Kong's Economic System\\
\bigskip
\begin{enumerate}
\item The complete economic policy system.
\bigskip
\item The high economic freedom.
\bigskip
\item The prosperity of finance, information, transportation, international trade.
\bigskip 
\end{enumerate}

\end{frame}
\subsection{Rise}
\begin{frame}
\frametitle{The rise of China's Economy}
The rise of China's Economy\\
An economic miracle.\\
\bigskip
\begin{figure}[h!]
\centering
\includegraphics[width=1\textwidth]{hk8.jpg}
\label{threadsVsSync}
\end{figure}
\end{frame}



\begin{frame}
\frametitle{The rise of China's Economy}
The rise of economic zones like Beijing, Shanghai and    Shenzhen. \\
\begin{figure}[h!]
\centering
\includegraphics[width=0.7\textwidth]{hk8.png}
\label{threadsVsSync}
\end{figure}
\end{frame}
\begin{frame}{The rise of China's Economy}
The impact on Hong Kong's economic status\\
\bigskip
From the perspective of the background of Hong Kong's rise\\
\bigskip
\bigskip
Shanghai financial center went to the weak
$\Longrightarrow$ the rise of Shanghai, Beijing and Shenzhen\\
\bigskip
\bigskip
The economic autarky situation, the planned economy
$\Longrightarrow$ The more open economy, the maket economy.\\
\bigskip
\bigskip
The lack of bridge between China and foreign countries
$\Longrightarrow$ The second largest international trade country.\\
 



\end{frame}
\subsection{The impact on Hong Kong's economic status}
\begin{frame}
\frametitle{The rise of China's Economy}
\textcolor{red}{The impact on Hong Kong's economic status}\\
\bigskip
From the perspective of the institutinal reasons\\

\bigskip
\begin{itemize}
\item Government's policy change.
\bigskip
\item Political institutional contradiction.
\bigskip
\item Different economy system on tax, trade, accounting policy.
\end{itemize}

\end{frame}

\begin{frame}{The rise of China's Economy}

\transwipe<1>
\textcolor{red}{The impact on Hong Kong's economic status}\\
\bigskip
From the perspective of the cultural reasons.\\
\bigskip
\begin{itemize}
\item The Hong Kong’s native consciousness.
\bigskip
\item The missing of British.
\end{itemize}


\end{frame}

\section{Conclusion}
\begin{frame}{Conclusion}

\bigskip
\bigskip

\begin{block}{\center{Conclusion}}
Pretend to have a conclusion.
\end{block}
\end{frame}
\begin{frame}
\transsplitverticalin<1>
\bigskip
\bigskip
\bigskip
\bigskip
\bigskip
\bigskip
\bigskip


\center{Thank}\\ 
\center{YOU}
\end{frame}
\end{document}
